% Options for packages loaded elsewhere
\PassOptionsToPackage{unicode}{hyperref}
\PassOptionsToPackage{hyphens}{url}
%
\documentclass[
]{article}
\usepackage{lmodern}
\usepackage{amssymb,amsmath}
\usepackage{ifxetex,ifluatex}
\ifnum 0\ifxetex 1\fi\ifluatex 1\fi=0 % if pdftex
  \usepackage[T1]{fontenc}
  \usepackage[utf8]{inputenc}
  \usepackage{textcomp} % provide euro and other symbols
\else % if luatex or xetex
  \usepackage{unicode-math}
  \defaultfontfeatures{Scale=MatchLowercase}
  \defaultfontfeatures[\rmfamily]{Ligatures=TeX,Scale=1}
\fi
% Use upquote if available, for straight quotes in verbatim environments
\IfFileExists{upquote.sty}{\usepackage{upquote}}{}
\IfFileExists{microtype.sty}{% use microtype if available
  \usepackage[]{microtype}
  \UseMicrotypeSet[protrusion]{basicmath} % disable protrusion for tt fonts
}{}
\makeatletter
\@ifundefined{KOMAClassName}{% if non-KOMA class
  \IfFileExists{parskip.sty}{%
    \usepackage{parskip}
  }{% else
    \setlength{\parindent}{0pt}
    \setlength{\parskip}{6pt plus 2pt minus 1pt}}
}{% if KOMA class
  \KOMAoptions{parskip=half}}
\makeatother
\usepackage{xcolor}
\IfFileExists{xurl.sty}{\usepackage{xurl}}{} % add URL line breaks if available
\IfFileExists{bookmark.sty}{\usepackage{bookmark}}{\usepackage{hyperref}}
\hypersetup{
  hidelinks,
  pdfcreator={LaTeX via pandoc}}
\urlstyle{same} % disable monospaced font for URLs
\setlength{\emergencystretch}{3em} % prevent overfull lines
\providecommand{\tightlist}{%
  \setlength{\itemsep}{0pt}\setlength{\parskip}{0pt}}
\setcounter{secnumdepth}{-\maxdimen} % remove section numbering

\author{}
\date{}

\begin{document}

\emph{\textbf{~}}

\emph{\textbf{Principles of dengue virus evolvability derived from\\
genotype-fitness maps in human and mosquito cells}}

\emph{\textbf{~}}

Patrick T. Dolan\textsuperscript{1,2‡}, Shuhei
Taguwa\textsuperscript{1‡}, Mauricio Aguilar Rangel\textsuperscript{1},
Ashley Acevedo\textsuperscript{2}, Tzachi Hagai\textsuperscript{3}, Raul
Andino\textsuperscript{2}*, Judith Frydman\textsuperscript{1}*

\textsuperscript{1} Stanford University, Department of Biology,
Stanford, CA 94305

\textsuperscript{2} University of California, San Francisco,
Microbiology and Immunology, San Francisco, CA

\textsuperscript{3} School of Molecular Cell Biology and Biotechnology,
George S. Wise Faculty of Life Sciences, Tel Aviv University, Tel Aviv
69978, Israel

~

\textsuperscript{‡} These authors contributed equally.

* Corresponding authors jfrydman@stanford.edu; raul.andino@ucsf.edu

\textbf{ABSTRACT}

Viral pathogens often jump between host species, leading the virus down
new host-specific selective trajectories. Arthropod-borne viruses, like
Dengue virus (DENV), must be transmitted between human and insect hosts
to continue their transmission cycle,
\href{https://paperpile.com/c/DR88ah/ONpBI}{(\emph{1})}. Using serial
passage in human and mosquito cell lines and ultra-deep population
sequencing, we examined in unprecedented detail how DENV populations
adapt to two distinct host cellular environments. Our analysis
highlights distinct patterns of selection placed on the DENV genome in
these two hosts, as well as general principles of viral population
dynamics. We quantify the contribution of beneficial and deleterious
mutations to overall fitness. Mapping sites of positive and negative
selection across the genome, we identify key gate-keeper mutations. Our
a suggest molecular mechanisms of adaptation that provide evolutionary
robustness to transmission bottlenecks and facilitate interhost
transmission. We propose the features of arbovirus evolution uncovered
here may suggest general principles for inter-host viral transmission.

\textbf{INTRODUCTION}

RNA viruses represent a major threat to public health and societal
well-being. The great evolutionary capacity of RNA viruses, driven by
high mutation rates, allows them to adapt to their hosts and overcome
selective pressures, such as antiviral drugs and natural barriers to
infection
\href{https://paperpile.com/c/REZjPf/9TgsM+8nuz1+tRYDR+L1Afo}{\textsuperscript{2--5}}.
Understanding the molecular mechanisms of viral adaptation can reveal
central aspects of host tropism and viral pathogenesis, as well as
uncover fundamental principles governing molecular evolution.
Arboviruses such as dengue (DENV), Zika (ZIKV), and chikungunya (CHIKV),
which alternate between vertebrate and insect hosts, are a significant
cause of disease globally. With half of the world's population exposed,
DENV alone causes approximately 100-400 million infections, with 0.5
million severe infections and 10,000 deaths annually
\href{https://paperpile.com/c/REZjPf/Nx7yi}{\textsuperscript{1}}.~They
also represent an important model system for exploring fundamental
aspects of virus evolution and inter-host transmission, as they must
constantly adapt to changing selective pressures.

Characterizing the genetic and evolutionary mechanisms underlying
arboviral adaptation to the distinct environments of each host is key to
understanding their emergence and spread. The vertebrate and
invertebrate hosts of arboviruses differ significantly in temperature,
cellular environment, and modes of antiviral immunity, which shape the
evolutionary landscapes of viruses. Several studies are beginning to
reveal how arboviruses relate to these alternative landscapes,
identifying mutations that confer increased fitness in each
host\href{https://paperpile.com/c/REZjPf/mlDyr+r0H6+erQn1+mYIl+uzXme+XqhpA}{\textsuperscript{6--11}}.
However, we lack a comprehensive picture of the alternative arbovirus
fitness landscapes defined by the human and insect host environments.

RNA viruses exist as a dynamic population of co-circulating mutant
genotypes surrounding a master
sequence\href{https://paperpile.com/c/REZjPf/fySKT}{\textsuperscript{12}}.
While the distribution and dynamics of minor alleles are thought to play
important roles in population fitness, adaptation and disease
\href{https://paperpile.com/c/REZjPf/kVUH4+ETeaC+lp4YG+iU0v6+f5vgZ+zRVIk+vDnrd}{\textsuperscript{13--19}},
technical limitations of sequencing resolution have restricted the
analysis of experimentally evolved virus populations to allele
frequencies greater than 1 in 100 or 1 in 1000. Recent developments in
high-accuracy sequencing approaches
\href{https://paperpile.com/c/REZjPf/bczFl+CObqp}{\textsuperscript{20,21}},
such as Circular Sequencing (CirSeq), which can detect alleles as rare
as 1 in 10\textsuperscript{6} in frequency, allow us to probe much
deeper into the spectrum of diversity in a viral population. The ability
to trace the evolutionary dynamics of individual alleles from their
genesis at the mutation rate to their eventual fate in a given
experiment, allows the description of viral fitness landscapes with
unprecedented detail.

Here, we use CirSeq to characterize DENV adaptation to human and
mosquito cells. By tracing individual allele trajectories for almost all
possible single nucleotide variants across the DENV genome, we estimated
the influence of positive and negative selection in shaping the
evolutionary paths of DENV in the distinct environments. Analysis of the
allele repertoire contributing to viral population fitness reveals the
cumulative role of low-frequency alleles during adaptation. Moreover, we
find that adaptation relies on host-specific beneficial mutations
clustered in specific regions of the DENV genome. These regions are
enriched in structural flexibility and are also sites of variation
across naturally occurring DENV and ZIKV strains. By uncovering genetic
and biophysical principles of DENV adaptation to its two hosts, our
analysis provides insights into flavivirus evolution and reveals
parallels between long and short-term evolutionary scales.

\textbf{RESULTS}

~

\textbf{Phenotypic characterization of DENV populations adapting to
human or mosquito cells.}

Two simple models could describe how arboviruses cycle between their
alternative host environments (Fig. 1a). First, the viral genome could
have overlapping host-specific fitness landscapes, thus reducing the
trade-offs involved in transmission. Alternatively, the virus may have
distinct host-specific landscapes with offset fitness maxima. To
characterize the relative topography of the adaptive landscapes of DENV
in vertebrate and invertebrate hosts (Fig. 1a), we performed an
evolution experiment (Fig. 1b). Clonal, plaque-purified dengue virus
(DENV serotype 2/16681/Thailand/1984) was serially passaged in the human
hepatoma-derived cell line Huh7 or the \emph{Aedes albopictus}-derived
cell line C6/36, for nine passages. To maintain a constant, effective
population size and to control the influence of drift due to population
bottlenecks, a viral inoculum of 5x10\textsuperscript{5} focus forming
units (FFU) from the previous passage was used to initiate the next
passage by infecting 6x10\textsuperscript{5} cells at an MOI of 0.1. To
distinguish host-specific versus replicate-specific events, we carried
out two independent and parallel passaging experiments in each cell line
(Series A and B, Fig. 1b).

The fitness gains associated with adaptation were assessed
phenotypically by measurements of virus titer (Fig. 1c and Extended Data
Fig. 1a), intracellular vRNA content (Fig. 1d and Extended Data Fig.
1b), and focus size and morphology (Fig. 1e and Extended Data Fig. 1c
and d) for each viral population in the passaged host cell. All of these
fitness measures increased over time for the passaged host, indicating
significant adaptive evolution throughout the experiment. We quantified
fitness trade-offs in parallel by carrying out the same measurements in
the alternative (by-passed) host cell. In agreement with previous
studies
\href{https://paperpile.com/c/REZjPf/Bai6p+a4AV4+knFQ3+Mq62m+4eMRo+r0H6+YjlCW}{\textsuperscript{7,22--27}},
passaging on one host cell line was accompanied by a concurrent loss of
fitness in the alternative host cell line (Fig. 1c and Extended Data
Fig. 1e). For instance, the human-adapted virus showed a uniform small
focus phenotype when plated on mosquito cells (Fig. 1f). In contrast,
mosquito-adapted populations formed fewer foci in human cells (Fig. 1f).
Mosquito-adapted populations exhibited a heterogeneous focus phenotype,
with small and large foci, suggesting they contain distinct variants
differentially affecting replication in human cells.

We further assessed the evolutionary trade-offs during host adaptation
by comparing the relative titers of all the passaged populations in both
the original and the alternative host cells, Huh7 and C6/36. We also
measured viral titers on human Huh7.5.1 cells, an Huh7-derived line
lacking the RIG-I antiviral signaling pathway, as well as human
hepatoma-derived HepG2 cells, and Africa Green Monkey epithelial-derived
Vero cells (Fig. 1g and Extended Data Fig. 1d). For each passage, viral
titers were normalized to that obtained in the adapted (original) host
cell line, to yield the efficiency of plating (EOP) (individual EOP
plots shown in Extended Data Fig. 1d). The large number of comparisons
in 2D space were visualized using an embedding technique that summarizes
the relative EOP as an approximate distance in two dimensions. This
approach is similar to the technique of antigenic cartography used to
describe antigenic evolution from pairwise measurements of antibody
neutralizing titers
\href{https://paperpile.com/c/REZjPf/jN2vZ}{\textsuperscript{28}}. The
movement of the sequenced viral populations (Fig. 1g, red and blue
lines) relative to the placement of the cell lines (grey circles)
reflects the change in relative titer between both cell lines.
Human-adapted viruses exhibited similar titers in human- and
primate-derived cell lines, resulting in their clustering separately
from the titers in mosquito-derived C6/36 cells. The movement of
C6/36-adapted populations (blue line) towards the C6/36 cell line
reflects the significant mosquito-specific adaptation in these
populations. Notably, RIG-I-deficient Huh7.5.1 cells placed intermediate
to the other primate and insect cell lines, emphasizing the role played
by innate immune signaling in the trade-offs associated with host
adaptation.

\textbf{~}

\textbf{Characterizing genotypic changes in adapting DENV populations.}

To determine the genotypic changes associated with host cell adaptation,
we subjected all viral populations to CirSeq RNA
sequencing\href{https://paperpile.com/c/REZjPf/CObqp+jDAqh}{\textsuperscript{21,29}},
With an error rate of less than 1 in 10\textsuperscript{6}, CirSeq
yielded an average of approximately
2x10\textsuperscript{5}-2x10\textsuperscript{6} reads per base across
the genome for each viral population in our experiments (Fig.
2a)\href{https://paperpile.com/c/REZjPf/bczFl+jDAqh}{\textsuperscript{20,29}}.
This depth permits the detection of alleles as rare as approximately 1
in 60,000-600,000 genomes (Fig. 2b).~ ~ ~~~

We next compared the mutational spectra in the sequenced populations
(Passage 7 shown in Fig. 2b). No mutations reached fixation in the
adapting populations, but many alleles rapidly increased in frequency
with passage number (Extended Data Mov. 1). Comparison of the allele
frequencies in the two independent passage series A and B revealed that
the replicate populations (Fig 2c \emph{i} or \emph{ii}) shared more
high-frequency mutations (defined here as 1\% allele frequency) than
populations passaged in different hosts, which shared no high frequency
mutations (Fig. 2c \emph{iii}). Human cell-adapted lineages shared
high-frequency mutations in NS2A and NS4B, while mosquito replicates
shared high-frequency mutations in E, NS3, and the 3′ UTR.

To better visualize the high-dimensional temporal dynamics of adaptation
(Extended Data Mov. 1), we employed two alternative dimension reduction
approaches that summarized the population sequencing data. Principal
components analysis (PCA) quantifies the patterns of allele frequency
variance between the populations, identifying independent patterns of
variance (Fig. 2d and Extended Data Fig. 2a and b). Multidimensional
scaling (MDS) examines the genetic distance and divergence of the
populations over time (Fig. 2e). Both analyses demonstrate that the
serially passaged DENV populations exhibit host-specific patterns of
variation, revealing host-specific changes in genotypic space over
passage in response to selection (Fig. 2d and e).

The PCA revealed the contribution of host-specific and
replicate-specific changes in the viral population. The first four
components of the PCA explained 96\% of the observed allele frequency
variance in the experimental populations. The first two components,
which explain 83\% of the observed variance (Fig 2d \emph{i} and
Extended Data Fig. 2a), partitioned the viral lineages along two
orthogonal, host-specific paths, radiating outward in order of passage
number from the original WT genotype (Fig. 2d \emph{i}). The third and
fourth components in the PCA explained 13.4\% of the observed variance
and further partition each lineage along orthogonal, replicate-specific
axes (Fig. 2d \emph{ii} for human series A and B and Fig. 2d \emph{iii}
for mosquito series A and B). The PCA-derived scores for individual
alleles in component space summarized their contribution to the host-
and replicate-specific dynamics (Extended Data Fig. 2b). The 3′ UTR and
E contained the strongest signatures of mosquito-specific adaptation.
Human-specific alleles were distributed across the genome, including
nonsynonymous substitutions in E, NS2A, and NS4B (Extended Data Fig.
2b). Replicate-specific PCA components also highlighted clusters of
alternative alleles in E and the 3' UTR.

Multidimensional scaling (MDS) allows the embedding of the
multidimensional pairwise distances between populations into two
dimensions, providing a complementary view of the genetic divergence of
the populations. Like PCA, this visualization also captures the
orthogonal host-specific adaptive paths traveled by the populations (Fig
2e). In general, passage also corresponded to a loss of diversity in the
population, measured as Shannon's entropy (Fig. 2e).

The finding that reproducible population structures emerge during DENV
adaptation to each host resonates with theoretical predictions that
large viral populations will develop equilibrium states, commonly
referred to as quasispecies, deterministically based on the selective
environment
\href{https://paperpile.com/c/REZjPf/Clw9O+rZbzc}{\textsuperscript{30,31}}.
The host-specific composition of these populations likely reflects the
differences in the selective environments that determine host range and
specificity, prompting us to dissect their composition further.

~

\textbf{Fitness landscapes of DENV adaptation to human and mosquito
cells.}

The concept of fitness links the frequency dynamics of individual
alleles in a population with their phenotypic outcome, i.e., beneficial,
deleterious, lethal, or neutral. Lethal and deleterious alleles are held
to low frequencies by negative selection, while beneficial mutations
increase in frequency due to positive selection (Fig. 3a). Observing the
frequency trajectory of a given allele over time relative to its
mutation rate enables the estimation of its fitness effect.

We took advantage of the resolution of CirSeq to estimate the
substitution-specific per-site mutation rates for DENV using a
previously described maximum likelihood (ML) approach (Extended Data
Fig.
3a)\href{https://paperpile.com/c/REZjPf/bczFl}{\textsuperscript{20}}.
These estimates, ranging between 10\textsuperscript{-5} to
10\textsuperscript{-6} substitutions per nucleotide per replication
(s/n/r) for each substitution, agreed well across populations. C-to-U
mutations occurred at the highest frequency, approximately
5x10\textsuperscript{-4} s/n/r in all populations, potentially
reflecting the action of cellular deaminases (e.g., APOBEC3 enzymes
\href{https://paperpile.com/c/REZjPf/M9gyT}{\textsuperscript{32}}). The
genomic mutation rate, substitutions per genome per replication (s/g/r)
(⎧\textsubscript{g}), was calculated by taking the sum of the ML
mutation rate estimates of all single-nucleotide mutations across the
genome, yielding ⎧\textsubscript{g} estimates of 0.70 and 0.73 s/g/r for
mosquito populations and 0.61 and 0.60 s/g/r for human populations,
consistent with genomic mutation rate estimates for other
positive-strand RNA viruses (Extended Data Table 3)
\href{https://paperpile.com/c/REZjPf/I2u2G}{\textsuperscript{33}}.

Using a model derived from classical population genetics previously
described in (Fig. 3b), we next generated point estimates and 95\%
confidence intervals of relative fitness (\(\widehat{w}\)) for each
possible allele in the DENV genome (Fig 3c and d). The distribution of
mutational fitness effects (DMFE) is commonly used to describe the
mutational robustness of a given genome (Fig. 3c and Extended Data Fig.
3b). Describing the full spectrum of the DMFE, including the large
subset of deleterious alleles, requires significant sequencing depth to
establish each allele's behavior relative to its mutation rate (Fig.
3b). The vast majority of alleles cannot be detected by clonal
sequencing or conventional deep-sequencing approaches (Fig. 3d and
Extended Data Fig. 3c), which can only detect a few high frequency
beneficial and neutral mutations. Due to its low error rate, CirSeq
reveals a much richer fitness landscape of adapting DENV populations
(Fig. 3d), enabling us to examine the role of lower frequency alleles,
such as deleterious mutations evolving under negative selection, on
virus evolution.

~The DMFEs of DENV exhibited bimodal distributions with peaks at
lethality (\(\widehat{w}\)=0) and neutrality (\(\widehat{w}\)=1.0), and
a long tail of rare beneficial mutations
(\(\widehat{w}\)\textgreater1.0), similar to what is observed for other
RNA viruses
\href{https://paperpile.com/c/REZjPf/bczFl}{\textsuperscript{20}}. The
95\% CIs of these \(\widehat{w}\) fitness estimates were used to
classify individual alleles as beneficial (\emph{B}), deleterious
(\emph{D}), lethal (\emph{L}), or neutral (\emph{N}) (Fig. 3e and
Extended Data Fig. 3a). Alleles with fitness 95\% CI maxima equal to 0
were classified as lethal alleles; these never accumulate above their
mutation rate due to rapid removal by negative selection (Fig. 3b and c,
black). Alleles with an upper CI higher than 0 but lower than 1.0 were
considered deleterious (Fig. 3b and Extended Data Fig. 3b, purple).
Alleles with a lower CI greater than 1.0 were classified as beneficial;
these accumulate at a rate greater than their mutation rate due to
positive selection (Fig. 3b and Extended Data Fig. 3b, yellow). Alleles
whose trajectories could not be statistically distinguished from neutral
behavior (\(\widehat{w}\) =1.0) are referred to as `neutral' (Fig. 3b
and e, and Extended Data Fig. 3b, gray)
\href{https://paperpile.com/c/REZjPf/bczFl}{\textsuperscript{20}}.

The genomic mutation rate represents the rate at which novel mutations
enter the population (Fig. 3f, "Total"). To understand the expected
fitness of new mutations, we used fitness classifications for all 32,166
possible single nucleotide variant alleles (Fig 3e) to estimate the
genomic beneficial, deleterious, and lethal mutation rates (Fig. 3f).
These estimates indicate that the virus maintains a significant
deleterious genetic load due to the high rate at which deleterious and
lethal mutations flow into the population. A given DENV genome will
accumulate a deleterious mutation in 40-50\% of replications, compared
to a 0.2-0.3\% chance of accumulating a beneficial mutation in any
replication (Fig 3f).

~

\textbf{Defining Constraints Shaping the DENV Fitness Landscape.}

We compared the proportion of mutations in each fitness class for
structural and non-structural regions of the viral polyprotein. A high
confidence set of 13-14,000 alleles in each population was chosen based
on sequencing depth and quality of the fit in the \(\widehat{w}\)
fitness estimates across passages. There were striking differences in
the distribution of lethal and deleterious mutations in distinct regions
of the genome (Fig. 3g). Non-structural proteins were significantly
enriched in deleterious and lethal mutations compared to structural
proteins. The increased mutational robustness of DENV structural
proteins compared to non-structural ones contrasts with poliovirus,
where structural proteins are less robust to mutation that nonstructural
proteins
\href{https://paperpile.com/c/REZjPf/bczFl}{\textsuperscript{20}}. These
differences may arise from the distinct folding and stability
constraints of the enveloped and non-enveloped virion structures. We
also find the viral UTRs exhibit host-specific patterns of constraint,
consistent with their host-specific roles in the viral life cycle
\href{https://paperpile.com/c/REZjPf/r0H6+delW7+4f1JN}{\textsuperscript{7,34,35}}.
In human cells, the DENV UTRs were more brittle but also contained more
beneficial alleles than in mosquito adapted populations, suggesting
strong selection.

In contrast to beneficial mutations, which were largely host and
replicate specific, deleterious and lethal mutations exhibited
significant overlap between the two hosts (Fig. 3h and Extended Data
Fig. 3d), indicating common constraints on viral protein and RNA
structures and functions in the two host environments. These biophysical
constraints were further examined by evaluating how specific mutation
types contribute to the viral fitness landscape (Fig 3i-k). As expected,
synonymous mutations tended to be more neutral than non-synonymous
mutations, which exhibited a bimodal distribution of fitness effects
(Fig 3k). We further partitioned non-synonymous mutations into
conservative substitutions (Fig. 3i-k, ``Cons.''), which do not
significantly change the chemical and structural properties of
sidechains, and non-conservative which do (Fig. 3i-k,
``Non-cons.'')\href{https://paperpile.com/c/REZjPf/EVbva}{\textsuperscript{36}}.
Non-conservative changes exhibited a significantly greater deleterious
fitness effects that conservative changes, emphasizing the impact of
biophysical properties on fitness effects as well as the sensitivity of
our approach to uncover these differences
\href{https://paperpile.com/c/REZjPf/EVbva}{\textsuperscript{36}}. As
expected, lethal alleles were enriched in nonsense mutations (Fig. 3j,
``Stop'') as well as nonsynonymous substitutions (Fig. 3j). These
findings reveal the structural biophysical constraints shaping the DENV
adaptive landscape and constraining viral diversity.

~

\textbf{Linking population sequence composition to experimental
phenotypes}

Examining the allele fitness distribution in DENV populations as a
function of passaging in human or mosquito cells revealed a `fitness
wave' occurring over the course of the adaptation experiment (Fig 4a;
Extended Data Fig 4a). In early passages, the population is dominated by
neutral and deleterious mutations that arise during the initial
expansion. In later passages, when rare beneficial mutations begin to
accumulate under positive selection, we observe a concurrent loss of
deleterious and neutral mutations, likely driven out by negative
selection. However, because most mutations arising during replication
are deleterious or neutral (Fig. 3f), their genetic load is never fully
purged from the viral populations.

The Fundamental Theorem of Natural Selection dictates that the mean
relative fitness of a population should increase during adaptation
\href{https://paperpile.com/c/REZjPf/rysMZ+WYtJW}{\textsuperscript{37,38}}.
Given the low probability of acquiring multiple mutations per genome per
replication cycle (Fig. 3f), individual alleles were treated as
independent of each other in our previous analysis of fitness. However,
linking the fitness of individual alleles to the dynamics of genomes,
and estimating the aggregate effect of individual mutations, requires
the estimation of haplotypes (\(W\)). To this end, we used the fitness
values and frequency trajectories of individual alleles to estimate the
aggregate changes in fitness in the populations over the course of
adaptation to either host cell (Fig. 4b). For each population, we
generated a collection of haplotypes, and corresponding genomic fitness
values (\(W\)) that reflect the empirical allele frequencies determined
by CirSeq (Fig 3). This simulated population was then used to explore
the changes in distribution of \(W\) at the population-level over the
course of passage. As expected, the median \(W\) of these simulated
populations increased throughout passaging in a given cell type (Fig.
4c), consistent with the dynamics of the individual constituent alleles
(Fig. 4a).

Next, we compared the genotype-based median fitness of the population
(Fig. 4d) with the experimental phenotype observed in the corresponding
viral population (Fig. 1). We chose mean absolute viral titers (Fig. 1c)
as a gross measure of replicative fitness and adaptation to the host
cell. We observed a striking correlation between viral titers and the
calculated median \(W\) of our simulated populations, which is based on
allele frequency trajectories alone (Fig 4d; R values ranging from 0.45
to 0.98). This correlation suggests that the estimates of population
fitness that incorporate the entire repertoire of cocirculating alleles
can predict the aggregate phenotype in these large populations.

We next examined the contribution of beneficial, lethal and deleterious
mutations to the population fitness, \(\), over the course of early
adaptation. To this end, we again calculated the distribution of genome
fitness for each adaptation lineage as described above, but taking into
account the influence of only beneficial, deleterious or lethal alleles.
Next, we compared these class-specific trajectories to the overall mean
population fitness (Fig. 4e, broken grey line) in order to estimate the
contribution of each class of mutation to the adaptation of the
populations over passage (Fig. 4e). Beneficial mutations, although
occurring relatively rarely, rapidly accumulate and drive the increase
in the mean relative fitness (Fig. 4e, yellow line). In contrast,
deleterious alleles, which individually are present at low frequencies
but occur on 40-50\% of genomes, contribute a significant mutational
load across all passages (Fig. 4e, purple line). Accordingly, the burden
of deleterious mutations reduces mean fitness by approximately 50\%
across all passages (Fig. 4e, purple line). Of note, lethal mutations
(Fig. 4e, black line) exert minimal effect on the population because
they are rapidly purged and remain very rare. These analyses reveal that
both beneficial and deleterious mutations play important roles in the
phenotypes of large, diverse viral populations. Importantly, they
indicate that the high levels of deleterious mutations in a viral
population exerts a persistent burden during adaptation. Sequencing
approaches like CirSeq will be essential for comparing the mutational
burden on different viral species to understand the balance between
mutational tolerance and constraint on viral genomes.

\textbf{~}

\textbf{Molecular and structural determinants of dengue host adaptation}

Analysis of the regions of the viral genome under positive and negative
selection in each host provided insights into the molecular mechanisms
of DENV inter-host adaptation. We calculated the mean fitness effect of
non-synonymous and noncoding mutations in 21 nucleotide windows and
mapped them onto the DENV2 genome (Fig. 5a). Regions of evolutionary
constraint, where most mutations are deleterious (purple) or lethal
(black), were found throughout the genome, and distributed similarly
between the two hosts. This likely reflect general constraints on
protein structure and function. For instance, regions in non-structural
proteins NS3, NS4B, NS5, and in the UTRs shared regions of strong
negative selection in both hosts, denoting key structural and functional
elements. In contrast, the patterns of positive selection along the
genome were distinct between the two hosts (yellow points in Fig. 5a).
Notably, many beneficial mutations were clustered at a few specific
locations in the genome (yellow points in Fig. 5a), suggesting
adaptation relies on host-specific selection at specific ``hotspots''.

To further analyze these hotspots, we mapped the allele fitness values
on the three-dimensional structure of dengue protein E, which forms the
outermost layer of the viral envelope (Fig. 5b and c). The clusters of
adaptive mutations identified in mosquito cells were under negative
selection in human-adapted populations (Fig. 5b). One mosquito-adapted
cluster mapped to residues 150-160, surrounding the glycosylation site
at N153 (Fig. 5c, zoomed region), which has been associated with
mosquito adaptation
\href{https://paperpile.com/c/REZjPf/hSNe}{\textsuperscript{39}}. Closer
examination of this loop (E152-155) revealed that different mosquito
alleles created alternative substitutions, N153D and T155I, with
identical phenotypic consequences, namely to abrogate N153
glycosylation. N153D eliminates the asparagine that becomes glycosylated
and T155I disrupts the binding of the oligosaccharyltransferase
mediating glycosylation (Fig. 5d). Since both mosquito-specific alleles
disrupt NxT glycosylation at this site
\href{https://paperpile.com/c/REZjPf/gTFia}{\textsuperscript{40}}, it
appears that eliminating this glycan moiety is beneficial in mosquitoes
but not in human cells (Fig. 5d). Interestingly, glycosylation pathways
diverge significantly between humans and insects, yielding very
different final glycan structures
\href{https://paperpile.com/c/REZjPf/JUNqO}{\textsuperscript{41}}. Since
this loop is a primary site of structural variation in E proteins of
dengue and related flaviviruses, including Zika virus
\href{https://paperpile.com/c/REZjPf/hrhvV}{\textsuperscript{42}}, its
diversification may reflect past cycles of host-specific selection
acting on this region of E.

The ability of CirSeq to detect alleles at frequencies close to the
mutation rate permits detection and quantification of negative
selection, revealing sites that are critical for viral replication.
Therapies targeting these highly constrained regions under strong
negative selection may be less susceptible to resistance mutations. For
instance, both the RNA polymerase and the methyltransferase active-sites
of NS5 are enriched in lethal mutations in residues contacting the
enzyme substrates (Fig. 5e). Further analysis of mutations in the
methyltransferase residues contacting its ligands SAM and mRNA cap
illustrates the power of our analysis. We find residues that contact the
ligands through sidechain interactions are under strong negative
selection, with most mutations highly deleterious.~ In contrast,
residues that interact with the ligands through backbone interactions
were relaxed in their fitness effects (Fig. 5f). Thus, such
high-resolution evolutionary analyses could complement structure-based
anti-viral drug design by identifying regions of reduced evolutionary
flexibility, which may be less prone to mutate to produce resistance.

Our analyses also captured key differences in the evolutionary
constraints on the viral 3′ UTR (in Fig. 5h). We find that stem-loop II
and the nearly identical stem-loop I in the 3′ UTR show significant
shifts in mutational fitness effects between human and mosquito cells
(Figs. 5h and i). These stem-loops are conserved across flaviviruses and
form a "true RNA knot," capable of resisting degradation by the
exonuclease
XRN1\href{https://paperpile.com/c/REZjPf/Zs1LR+b9GgS+o9YCc}{\textsuperscript{43--45}}.
While previous findings indicated that in mosquito DENV acquired
deletions in these loops
\href{https://paperpile.com/c/REZjPf/r0H6}{\textsuperscript{7}} our
findings indicate DENV also accumulates highly beneficial mutations that
alter stem-loop folding in mosquito, which may result in the
accumulation of distinct small RNAs upon XRN1 exonucleolytic activity.
Our ability to identify subtle shifts in the DMFE can thus point to
molecular mechanisms of selection and adaptation.

~

\textbf{Defining design principles of dengue virus adaptability}

The surprising finding that adaptive mutations cluster in specific
regions of the genome suggests that adaptation operates through discrete
elements. We next examined the structural and functional properties of
these elements to better understand the design principles of DENV
evolution. The dengue polyprotein consists of soluble and transmembrane
domains. We found that transmembrane domains were depleted of beneficial
mutations and enriched in lethal mutations (Fig 6a). Thus, despite
differences in lipid composition of human and insect membranes
\href{https://paperpile.com/c/REZjPf/XAKft+zRdX7}{\textsuperscript{46,47}},
the transmembrane regions of DENV disfavor change during host cell
adaptation. For non-transmembrane DENV regions, we found striking
differences between structured domains and intrinsically disordered
regions (IDRs) (Fig 6b).~ Beneficial mutations were highly enriched in
IDRs, but not in structured regions (Fig. 6b). In contrast, lethal
mutations were enriched in ordered domains, while strongly depleted from
IDRs, highlighting the evolutionary constraints imposed by maintaining
folding and structure. Interestingly, this suggests that adaptation to
each host cell may be facilitated by variation in flexible,
surface-exposed disordered regions. Because IDRs have few structural
constraints and tend to mediate protein-protein and protein-RNA
interactions {[}add Citations{]}, their role as potential hotspots for
the evolutionary remodeling of virus-host-specific networks
\href{https://paperpile.com/c/REZjPf/mhPR+KHg7}{\textsuperscript{48,49}}.

We next examined whether the patterns of selection we observed in our
short-term adaptation study relate to patterns observed in long-term
evolution of flaviviruses (Fig. 6c). Sequence alignments of all four
major DENV serotypes were used to classify amino acid residues that are
invariant across the four strains, those with conservative substitutions
that maintain chemical properties, and those with highly variable
non-conservative substitutions. Strikingly, when compared to the fitness
classes derived in our study, the occurrence of lethal, detrimental, and
beneficial mutations mirrored the evolutionary conservation and variance
across DENV serotypes (Fig. 6c). For instance, beneficial mutations in
our study were strongly enriched in the regions of highest variation
across DENV serotypes, while lethal mutations were enriched in residues
that are invariant during evolution. These conclusions were supported
when we extended this analysis to include both DENV and Zika virus
isolates (Fig. 6d), suggesting a conserved core of residues that shape
the boundaries of the flavivirus fitness landscape. This suggests that
our simple cell culture paradigm recapitulates patterns of negative and
positive selection that shape long-term evolution of DENV as it cycles
between human and mosquito hosts.

~

\textbf{DISCUSSION}

The ability of some viruses to switch or alternate between hosts often
underlies the emergence and transmission of human disease. To gain
insight into the principles of virus adaptation to different hosts, we
explored the global dynamics and fitness changes in DENV populations
adapting to human and mosquito cells. Using high-resolution sequencing,
we quantify the contribution of beneficial and deleterious mutations
shaping the evolutionary paths of DENV populations responding to
host-cell specific selective pressures. Our analysis shows that DENV
populations move in largely reproducible paths during adaptation to
human or mosquito cells, driven by host-specific selection (Fig. 2). The
host-specific population structures that emerge as a result reflect the
unique genotype-fitness landscapes defined by each host cell (Figs. 1
and 2). As the population shifts in sequence space and result in a
concurrent increase in phenotypic fitness, as assessed by focus
morphology and absolute titers (Fig. 1). Strikingly, simple models that
take into account the estimated fitness and frequency dynamics of all
mutations in the population, derived from sequencing alone, can predict
these phenotypic changes in the population (Fig. 4). Decomposing the
evolutionary dynamics of the adapting virus population further, we can
estimate the contributions of any set of mutations on the population. An
insight of these analyses is our ability to quantify the large burden of
detrimental mutations that persists in viral populations throughout
passage, imposing a significant fitness cost. This observation resonates
with the finding that the high mutation load of RNA viruses, while
driving adaptation, places them at the ``error-catastrophe threshold''
\textsuperscript{3, 4, 14}. It will be interesting to use these datasets
to further deconstruct the contributions of specific regions of the
genome this load across viral species. For example, in comparing our
findings in DENV with our previous studies in poliovirus, we find that
structural proteins contribute a much greater proportion of the
deleterious mutational load on the genome of poliovirus than in the case
of DENV.

We find adaptive mutations are clustered in specific coding and
noncoding regions across the DENV genome. Examination of
mosquito-specific alterations in a glycosylation site in protein E and
in the 3' UTR suggest that mutations in these clusters correspond to
competing mutations with similar phenotypic outcomes. For instance,
mutations that cluster in the 3′ UTR change the structure of stem-loop
II, a known site for gate-keeper mutations for mosquito transmission
\href{https://paperpile.com/c/REZjPf/r0H6+mYIl}{\textsuperscript{7,9}}.
Similarly, mutations that cluster in loop 150-160, including the two
dominant substitutions at 152 and 155 in DENV E protein disrupt
glycosylation in mosquito cells
\href{https://paperpile.com/c/REZjPf/hSNe}{\textsuperscript{39}}. This
suggests that the process of host adaptation relies on similar adaptive
phenotype. We propose that the phenotypic redundancy of such mutations
increases the mutational target size associated with key transitions
necessary for adaptation, thereby partially relieving possible
bottlenecks associated with transmission and early adaptation.~

Our study uncovers the role of structural constraints in shaping DENV
evolution. Adaptive mutations are largely excluded from transmembrane
domains and structured regions in DENV proteins. Thus, structural
integrity places significant constraints on variation within these
regions. It is tempting to speculate that the sequence of these
arboviral domains is poised at a compromise that optimizes function in
the distinct environments of human and mosquito cells
\href{https://paperpile.com/c/REZjPf/KuSMk+gqelx+QZFTJ}{\textsuperscript{50--52}}.
Of note, highly constrained DENV regions, where all mutations are
deleterious or lethal, may be informative for antiviral or vaccine
design. Beneficial mutations, likewise, are enriched in flexible loops
and intrinsically disordered regions of the DENV polyprotein (Fig 6b).
The relaxed structural constraints of IDRs allow them to explore more
mutational diversity without compromising protein folding or stability,
thus enabling access to more extensive sets of adaptive mutations
\textsuperscript{48,53}. Such plasticity may allow viral IDRs to rewire
viral protein interactions with host factors, thereby driving adaptation
to changing environments (Fig 6f).~~

Notably, the link between structural properties and fitness effects
measured in our study mirrors sequence conservation and variation across
natural isolates of DENV and ZIKV (Fig. 6c, d). This indicates that the
relationships between adaptability, structural flexibility, and
phenotypic redundancy uncovered here for DENV adaptation to cultured
human and mosquito cells can inform on general principles in flavivirus
evolution. While arbovirus that cycle between human and mosquito
represent a more extreme case of host switching, most emerging viruses
must adapt to changing environments during zoonotic transmission or
intra-host spreading. We propose that our simple experimental approach
can identify regions mediating evolutionary changes linked to
diversification, tropism, and spread for a wide range of RNA viruses.

\textbf{METHODS}

\textbf{Cells}

Huh7, Huh7.5.1, HepG2, Vero cells were cultivated at 37˚C and C6/36
cells at 32˚C, respectively, as described previously
\href{https://paperpile.com/c/REZjPf/py7tY}{\textsuperscript{54}}.

\textbf{Viruses}

DENV2 strain 16681 viral RNA generated by \emph{in vitro} transcription
was electroporated to produce progeny virus as described in
\href{https://paperpile.com/c/REZjPf/py7tY}{\textsuperscript{54}}. For
passaging, DENV of 5 x 10\textsuperscript{5} FFU were serially
propagated in one 10cm dish of Huh7 or C6/36 cells. The culture medium
was collected before the cells showed a severe cytopathic effect (CPE).
At each passage, virus titers in the supernatant were measured and
adjusted for the next passage.

\textbf{Focus forming assay}

Semi-confluent cells cultured in 48-well plates were infected with a
limiting 10-fold dilution series of virus, and the cells overlaid with
culture medium supplemented with 0.8\% methylcellulose and 2\% FBS. At 3
(Huh7) or 4 (C6/36) days post-infection, the cells were fixed by 4\%
paraformaldehyde-in-PBS, stained with anti-E antibody and visualized
with a VECTASTAIN Elite ABC anti-mouse IgG kit with a VIP substrate
(Vector Laboratories, Burlingame, CA USA). The entire wells of 48-well
plates were photographed by Nikon DSLR camera D810, and each foci size
was measured by image-J. Each experiment was performed in duplicate.

\textbf{Quantitative Real-Time PCR (qRT-PCR)}

The intracellular RNAs were prepared by phenol-chloroform extraction.
cDNA was synthesized from purified RNA using the High Capacity cDNA
Reverse Transcription Kit (Life Technologies), and qRT-PCR analysis
performed using gene-specific primers (iTaq\textsuperscript{TM}
Universal Supermixes or SYBR-Green, Bio-Rad) according to manufacturers'
protocols. Ct values were normalized to GAPDH mRNA in human cells or 18S
rRNA in mosquito cells. qRT-PCR primers are listed in Table S1. Each
experiment was performed in triplicate.

\textbf{CirSeq and Analysis of allele frequencies}

For preparing CirSeq libraries, each passaged virus (6 x
10\textsuperscript{6} FFU) was further expanded in parental cells seeded
in four 150 mm dishes. The culture medium was harvested before the
appearance of severe CPE, and the cell debris was removed by
centrifugation at 3,000 rpm for 5 min. The virion in the supernatant was
spun down by ultracentrifugation at 27,000 r.p.m, 2 hours, 4ºC and viral
RNA was extracted by using Trizol reagent. Each 1µg RNA was subjected to
CirSeq libraries preparation as described previously
\href{https://paperpile.com/c/REZjPf/CObqp}{\textsuperscript{21}}.
Variant base-calls and allele frequencies were then determined using the
CirSeq v2 package.

\textbf{Calculation of Relative Fitness}

An experiment of \(N\) serial passages will produce, for any given
single nucleotide variant (SNV) in the viral genome, a vector \(X\) of
variant counts at each passage,\(\text{\ t}\):

\[X = \{ x_{1},\ldots,\ x_{t},\ \ldots,x_{N}\}\]

And, a vector Y containing the corresponding coverages at each passage,
\(t\):

\[Y = \left\{ y_{1},\ldots,\ y_{t},\ \ldots,y_{N} \right\}\]

As explained previously\textsuperscript{1}, the relative fitness of a
SNV, \(w\), at time \(t\) can be described by the linear model:

\[\frac{x_{t}}{y_{t}} = \frac{x_{t - 1}}{y_{t - 1}}{*w}_{t}\  + \ \mu_{t - 1}\ \ \ (1)\]

Where \(\mu_{t - 1}\) is the estimated mutation rate for the variant at
time \(t - 1\) (described previously\textsuperscript{1}). This model
requires only two consecutive passages to estimate a relative fitness
parameter. However, to account for and quantify passage-to-passage noise
in the estimates of relative fitness we used the values of \(w\) across
passages to estimate the mean and variance of \(w\) for each SNV.

To account for genetic drift in our experiment, we used a similar
approach as Acevedo et al., 2014. At each passage, a fixed number of
focus forming units, \(\beta,\) are used to infect each subsequent
culture. In each \(\beta\) virions, \(b_{t - 1}\) of them will carry a
given SNV. Therefore, \(\frac{b_{t - 1}}{\beta}\) can be used to express
the frequency of that SNV in the transmitted population\(,\) which when
substituted for the term, \(\frac{x_{t - 1}}{y_{t - 1}}\), in the right
side of equation (1) will yield:

\[\frac{x_{t}}{y_{t}} = \frac{b_{t - 1}}{\beta}{*w}_{t}\  + \ \mu_{t - 1}\]

or:

\[w_{t} = \frac{\beta(\frac{x_{t}}{y_{t}}\  - \mu_{t - 1})}{b_{t - 1}}\ \ \ (2)\]

where:

\[b_{t - 1}\ \sim\ B(\frac{x_{t - 1}}{y_{t - 1}},\ \beta)\ \ \ (3)\]

Given that \(\beta\) is constant (5x10\textsuperscript{5} FFU), we need
only calculate \(b_{t - 1}\) in order to estimate \(w_{t}\) values.
Since we do not know the real value of \(b_{t - 1}\) for any variant,
especially for low frequency variants which are sensitive to
bottlenecks, we need to estimate it. This can be done by sampling \(m\)
times from equation (3). Such sampling is described by a Poisson
distribution, then:

\[\arg\ \max\ \frac{\lambda^{k}}{k!}e^{- \lambda}\ \ \ \ (4)\]

will give a maximum likelihood estimate for \(\lambda{= b}_{t - 1}\),
while the upper bound of \(k\) is given by \(\beta\). Doing so for each
\(x\) from time 1 to N-1 of gives a vector, \(\overline{B}\), of \(b\)
values: \(\overline{B} = \{ b_{1,}b_{2,\ }\ldots\ ,\ b_{N - 1\ }\}\).
Finally we estimate N-1 \(w\) values by solving equation (2) using each
element of \(\overline{B}\). This gives a vector
\(\overline{W} = \{ w_{t_{1}}\ldots,w_{t_{N - 1}}\}\).

Then­, the slope of the linear regression over the cumulative sum of
\(\overline{W}\) yields the estimated relative fitness, \(\widehat{w},\)
of a given SNV.­­ For this regression, we employed the Thiel-Sen
regression method (implemented in the R package `deming'
\textsuperscript{55}), given that some of our vectors \(\overline{W}\)
contains outliers as the result of \(\overline{X}\) having zeros due to
poor coverage. This regression will allow for the estimate to be robust
to those outliers, to avoid classifying them as detrimental variants
because spurious zeros. At the same time, for \(\overline{W}\) with a
majority of zeros and some positive observations, that are likely to
come from elements in \(X\) that are not significant (i.e. sequencing
errors), the Thiel-Sen estimate will give more weight to the real zero
values, classifying them as lethal or deleterious, and neglecting the
effect of the positive elements in \(W\). Finally, we also obtain an
estimate of the 95\% confidence interval by the procedure described
previously
\href{https://paperpile.com/c/REZjPf/xDHhy}{\textsuperscript{56}} and
implemented in the `deming' package
\href{https://paperpile.com/c/REZjPf/TJLiW}{\textsuperscript{55}}.

\textbf{Calculation of Mean Fitness}

To estimate the effect of the observed population dynamics on the
fitness of individual viral genomes and the distribution of genome
fitness in the absence of haplotypic information, we generated a
population of genomes sampled from the empirical allele frequencies. For
each genome, sequence identity and corresponding fitness
estimates,\(\ \widehat{w}\), at each position were randomly selected
with a probability equal to its frequency in each sequenced population.
Following sampling, the fitness of each genome in the sampled population
was calculated as the product of the \(\widehat{w}\) values across all
positions:

\(W = \prod_{i = 1}^{n}{\widehat{w}}_{i}\) (5)

For all simulations, 50,000 genomes were sampled, and statistical
analyses performed on the resulting populations; similar results were
obtained with independent samples. To estimate the contribution of
mutations in the individual fitness classes (i.e. beneficial,
deleterious, or lethal) to the aggregate population fitness, sampling
was performed as described above. Following sampling, the genome
fitness, \(W\), was calculated as described in equation (5), but only
taking into account substitutions of a given fitness class.

\textbf{Dimension reduction of genotypic data.} Principal components
analysis was performed on the unscaled population allele frequencies
using the `princomp' function in the R base
\href{https://paperpile.com/c/REZjPf/uGyfe}{\textsuperscript{57}}.
Calculation of Reynold's Θ was performed using the adegenet
\href{https://paperpile.com/c/REZjPf/V25w7}{\textsuperscript{58}} and
poppR \href{https://paperpile.com/c/REZjPf/TEfJA}{\textsuperscript{59}}
packages in R
\href{https://paperpile.com/c/REZjPf/uGyfe}{\textsuperscript{57}}.
Classical MDS (by Torgerson scaling
\href{https://paperpile.com/c/REZjPf/0m4TS}{\textsuperscript{60}}) was
performed to embed the pairwise Reynolds distances (Θ)
\href{https://paperpile.com/c/REZjPf/8bj1M}{\textsuperscript{61}}
between the viral populations in 2-dimensions.

\textbf{Dimension Reduction of Phenotypic Data}. Stress Minimization by
Majorization (implemented in \emph{SMACOF
\href{https://paperpile.com/c/REZjPf/gUxkc+dJiR3}{\textsuperscript{62,63}}})
was used for the ordination of cells and viruses based on empirical
relative titer data. The input distance matrix was generated from the
mean of log\textsubscript{10} titer measurements for (N=4) focus
formation assays on each passaged population on each of five cell lines:
Huh7, Huh7.5.1, C6/36, HepG2, and Vero. Titer values were
log\textsubscript{10} transformed and subtracted from the maximum
log\textsubscript{10}(titer) for each cell line to yield a matrix of
Cell-to-Population distances, where the minimum distance represents the
highest relative viability for each virus population.

\textbf{Structural Analysis}

Fitness values for non-synonymous mutations were displayed on available
dengue pdb structures using pyMol2 (Schrödinger). Data was aligned to
structures using in-house scripts. Briefly, protein sequences for each
chain in the PDB structure are mapped to the dengue 2 reference
polyprotein sequence using the Smith-Waterman algorithm for pairwise
alignment (implemented in `SeqinR' package). To emphasize regions of
positive selection, the values displayed on the structures represent the
lower 95\% confidence limit of the fitness estimate. Where multiple
non-synonymous alleles could be mapped to a single residue, the maximum
of the lower 95\% confidence limits were displayed to emphasize the most
significantly positively selected alleles at any position.

\textbf{Biophysical properties analyses:}

We have identified transmembrane regions using TMpred
\href{https://paperpile.com/c/REZjPf/pdAby}{\textsuperscript{64}},
taking regions with a score above 500 as \emph{bona fide} transmembrane
regions. For disorder prediction, we used IUPred2A
\href{https://paperpile.com/c/REZjPf/GYlhT}{\textsuperscript{65}}, using
the ``long'' search mode with default parameters. We took residues with
a value \textgreater{} 0.4 to be disordered. We used Anchor from the
same IUPred2A package, to find regions within disordered regions that
likely harbor linear motifs, using the default Anchor parameters and
taking residues with a score \textgreater{} 0.4 to be part of
motif-containing regions. For each of these regions (TM, non-TM,
ordered, disordered, and motif-embedding disordered regions), we have
computed the fraction of non-synonymous mutations that belongs to each
mutation category (beneficial, neutral, deleterious and lethal). We then
compared these to the respective fractions of the four categories in
non-synonymous mutations across the entire polyprotein. We used a
one-sided Fisher exact test to test for enrichment (or depletion) in
each of the biophysically-defined regions, in comparison with the entire
polyprotein, and adjusted the p-values using the Benjamini-Hochberg
\href{https://paperpile.com/c/REZjPf/lMDMS}{\textsuperscript{66}}
correction. Fig 6 shows the significance of these tests. We plot the
relative enrichment for different categories of mutations across
different biophysical regions. Relative enrichment is computed as the
difference between the fraction of occurrence in the tested region and
the fraction of occurrence in the entire polyprotein, divided by the
occurrence in the entire polyprotein. For example, relative enrichment
of lethal mutations in TM region is calculated as:
\(\frac{{\text{fraction}_{B}}_{\text{TM}} - \ {\text{fraction}_{B}}_{\text{PP}})}{{\text{fraction}_{B}}_{\text{PP}}}\)

\textbf{Cross viral strain and species analysis:}

We have aligned and compared the conservation of each residue in the
polyprotein of the dengue strain we used (strain 2) with DENV1, 3, and 4
serotypes (REF) using CLUSTAL W
\href{https://paperpile.com/c/REZjPf/gK8Xk}{\textsuperscript{67}}. We
extracted from the multi-sequence alignment the residues that are
conserved across the four species (identical), residues that are
substituted by a similar residue, and residues that have dissimilar
substitutions or gaps. We then compared the distribution of mutations
from the four categories, based on our experimental data analysis
(beneficial, neutral, deleterious, and lethal mutations) with their
distribution across the entire polyprotein. This was carried out as
described in "\textbf{Biophysical analysis}."

\textbf{{Figure Legends}}

\textbf{Fig. 1. Dengue navigates distinct fitness landscapes in its
alternative hosts.}

(a) Two potential models of the genotype-fitness landscape and
mutational network in alternative arboviral hosts. The relative
topography of the viral genotype-fitness landscape determines the extent
of evolutionary trade-offs associated with transmission, and the paths
through the mutational network toward host adaptation, and the
proportion of genotypes viable in the alternative host environment (grey
nodes).

(b) Outline of our \emph{in vitro} DENV evolution experiment. Dengue
virus RNA (Serotype 2/16881/Thailand/1985) was electroporated into
mosquito (C6/36) or human cell lines (Huh7), and the resulting viral
stocks were passaged at fixed population size (MOI = 0.1, 5E5
FFU/passage) for nine passages in biological duplicates. After passage,
viruses were characterized for phenotypic measures of adaptation.

(c) Viral production assays comparing mosquito-adapted (top panel) and
human-adapted (bottom panel) DENV populations. Adapted populations show
increased virus production on their adapted hosts. Biological replicate
A is shown for all experiments.

(d) Analysis of viral RNA content by qRT-PCR. Cellular DENV RNA is
significantly decreased when adapted lines are propagated on the
by-passed, alternative host. Lines and shading represent the mean and
standard deviation of 4 technical replicates, respectively\textbf{.}
Biological replicate A is shown for all experiments.

(e) Focus forming assays of the adapted populations over passage. Focus
size increased markedly throughout passage on the adapted host.

(f) Focus assays of the P9 virus on the adapted (left) and by-passed
(right) host. Changes in focus size and morphology suggest evolutionary
trade-offs between the alternative hosts.

(g) \emph{Tropic cartography} of DENV \emph{in vitro} host adaptation. A
two-dimensional embedding of the relative titer (mean of 4 replicates),
or efficiency of plating (EOP), of the adapted populations (red and blue
trajectories) relative to 5 assayed cell lines (grey points). Movement
corresponds to a change in EOP over passage on a log scale. The size of
the red and blue points indicate the ratio of the populations' titer on
the adapted and bypassed cell lines to illustrate the change in EOP over
passage on the adapted lines.

\textbf{Fig. 2. Adapting viral lineages show host-specific patterns of
genetic variance.\\
}(a) Adapted viral populations were subject to genotypic
characterization by ultra-deep sequencing using the CirSeq procedure.
(b) Plots of allele frequency across the viral genomes for all four
viral populations at passage 7. Alleles are colored by mutation type
(Nonsynonymous, Orange; Synonymous, Green; Mutations in the untranslated
region (UTR), Dark Grey). Shaded regions denote mature peptide
boundaries in viral ORF. (c) Scatter plots comparing allele frequencies
between adapted populations of human- and mosquito-adapted dengue virus.
Replicate host-adapted populations share multiple high-frequency
non-synonymous mutations, but populations from alternative hosts do not
(grey square, \textgreater10\%). (d) Dimension reduction of the allele
frequencies by principal components analysis summarizes the
host-specific patterns of variance (left), and the replicate-specific
differences in genetic variability over passage (right). (e) A
two-dimensional embedding of the pairwise genetic distances between the
sequenced viral populations (Weir-Reynolds Distance) by multidimensional
scaling. The viral populations (red and blue trajectories) project out
from the founding genotype in orthogonal and host-specific directions.
The size of the points corresponds to the diversity (computed as
Shannon's Entropy).

\textbf{Fig. 3. The distribution of fitness effects reveals patterns of
evolutionary constraint} (a) The frequency trajectories for G-to-A
mutations in the adapting populations determined by CirSeq. Colors
represent the classification of each allele as beneficial, deleterious,
lethal, or neutral (not statistically distinguishable from neutral
behavior) (b) Schematic illustrating the expected frequency behavior of
specific fitness classes relative to their corresponding mutation rate,
\(\mu\). Changes in allele frequency between passages are used to
estimate the fitness effects of individual allele­­s in the population
(see Methods). (c) Histogram showing the distribution of mutational
fitness effects (DMFE) of DENV passaged in mosquito cells. The data
shown are from mosquito A and represent the high confidence set of
alleles (see text). The fitness classifications of alleles in each bin,
based on their 95\% confidence intervals, is indicated by the fill
color. (d) The relative density of each mutation type across the fitness
spectrum illustrates the sequencing depth necessary to observe regions
of the fitness spectrum. Fill color represents the average frequency of
the mutation over passage. (e) Tabulation of all alleles by fitness
class. (f) Estimate of the genomic mutation rate per genome per
generation ("Total"), and fitness class-specific mutation rates ('B',
'D', 'N', and 'L'). (g) Area plot showing the fitness effects associated
with mutations in structural, non-structural, and UTR regions of the
DENV genome. The relative width of the columns indicates the number of
alleles in each class, the relative height of the colored regions
indicates the proportion of alleles of a given class. (h) Venn diagrams
showing the number of mutations identified as beneficial, deleterious,
or lethal in the high confidence set alleles (see methods). (i)
Histograms of the distribution of mutational fitness effects (DMFE)
broken down by mutation type. (j) Density plot of the relative density
of mutation types across the DMFE to emphasize the local enrichment of
specific classes. (k) Violin plots showing the relative fitness of
nonsynonymous and synonymous mutations, and those in the viral UTRs.
Nonsynonymous mutations can further be partitioned into conservative and
non-conservative classes, which differ significantly in fitness effect.

\textbf{Fig 4. Connecting global evolutionary dynamics and population
fitness}

(a) The surface plot of the 'fitness wave' illustrates the change in
frequency of alleles in each fitness bin throughout a passage
experiment. The height of the surface represents the sum of frequencies
of alleles in a given bin. Deleterious and neutral mutations (purple and
grey regions) make up a large proportion of the population early in the
experiment, but are largely driven out by beneficial mutations (yellow)
in later passages. (b) The aggregate effect of this fitness wave is an
increase in the mean population fitness, a weighted average of the
fitness of individual genomes in the population. To estimate the
distribution of genome fitness in the sequenced population in the
absence of haplotype information, we estimated a collection of plausible
haplotypes by sampling from the empirical frequencies of individual
mutations. We then estimated the fitness of each assembled genome as the
product of \(\widehat{w}\) estimates across all positions. (c) The
median and interquartile range of genome fitness (\(W\)) of 50,000
simulated genomes from each sequenced population. (d) Correlation plots
comparing the median genome fitness (\(W\)) of the sampled populations
and mean virus titer correspond well to each other. (e) Line plots
showing the influence of beneficial (yellow), deleterious (purple), and
lethal (black) mutations on the mean fitness of the population (grey
line). The shaded area represents the 95\% confidence interval of the
mean from 50,000 simulated genomes (see Materials and Methods).

\textbf{Fig. 5. Structural analysis reveals hotspots of viral
adaptation.}

(a) Barplot of the mean fitness effect of alleles in 21 nt windows
across the DENV genome. Synonymous alleles within the coding region are
removed to emphasize the fitness effects of nonsynonymous changes.
Yellow points above each line denote the locations of beneficial
mutations. Larger, labeled yellow points denote beneficial mutations
identified in both replicates of host cell passage. (b) The empirical
fitness estimates displayed on a trimer of envelope proteins in an
antiparallel arrangement, similar to that found on the mature virion. To
emphasize rare sites of positive selection, the color of each residue
represents the maximum of the lower confidence intervals of fitness
effect estimates at that site. Human-adapted populations show
significant negative selection on the envelope protein surface. (c)
Mosquito-adapted DENV exhibit two patches of pronounced positive
selection on the exterior face of the virion (labeled 120-130, and
150-160). These clusters are absent from the human-adapted populations.
Cluster \emph{150-160} (Zoom), consists of a loop containing a
glycosylation site at N153. (d) Plots of the frequency of local read
haplotypes for the area overlapping N153 and T155. Mutations at N153
(N153D) and T155 (T155I) are positively selected in mosquitoes, but
never occur together on individual reads. (e) Schematic describing the
phenotypically equivalent effects of the N153D and T155I mutations.
These mutations block recognition and modification by the host
oligosaccharyltransferase (OST), which initiates glycosylation. (f)
CirSeq also reveals patterns of negative selection. Patches of
significant evolutionary constraint can be seen around the
methyltransferase active site highlighted by numerous positions with
lethal fitness effects (Zoom). (g) Comparison of fitness effects of
non-synonymous mutations targeting residues in NS5-MT that interact with
the 5 (h) Insights into host-specific RNA structural constraints. Violin
plot comparing the fitness effects of mutations in the stem-loop (SL)
and dumb-bell (DB) structures of the 3′ UTR RNA of DENV2 shown in the
schematic. Fitness effects of mutations in the conserved structures
reveal differences in fitness effects associated with SLI and SLII in
human- and mosquito-adapted dengue virus populations. (i)
Nucleotide-resolution map of fitness effects on the viral 3′ UTR reveals
regions of SL1 and 2 that are under tighter constraints in human
passage.

\textbf{Figure 6: Biophysical and biological themes in DENV adaptation.}

(a, b) Distribution of mutations in regions with different biophysical
characteristics. (a) The relative enrichment of each type of mutations
(beneficial, deleterious, and lethal) in transmembrane (TM) regions
versus non-TM regions (Fig. 2a), and disordered versus ordered regions
(Fig. 2b). Relative enrichment is computed as the normalized difference
in occurrence of this type of mutation in the specific region tested,
and its occurrence across the entire polyprotein. Significance values
(FDR-corrected, Fisher exact values) are shown (* = p\textless0.05, ** =
p\textless0.01, *** = p\textless0.001).

(c) Cross-dengue conservation. Distribution of mutations in regions with
different levels of conservation across dengue virus strains. The
relative enrichment of each type of mutations (beneficial, deleterious
and lethal) in residues that are identical ("Invariant"), similar
("Conserved") or dissimilar ("Variable") across the four dengue strains.
Relative enrichment was calculated s computed as the normalized
difference in occurrence of this type of mutation in the specific region
tested, and its occurrence across the entire polyprotein. Significance
values (FDR-corrected, Fisher exact values) are shown (* =
p\textless0.05, ** = p\textless0.01, *** = p\textless0.001). (d)
Zika-Dengue conservation. Distribution of mutations in regions with
different levels of conservation between DENV and ZIKV virus. The
relative enrichment of each type of mutations (beneficial, deleterious,
and lethal) in residues that are identical, similar, or dissimilar
between the two viruses. Relative enrichment was calculated as the
normalized difference in occurrence fraction of this type of mutation in
this specific region and its occurrence across the entire polyprotein.
Significance values (FDR-corrected, Fisher exact values) are shown (* =
p\textless0.05, ** = p\textless0.01, *** = p\textless0.001). (e)
Visualization of a simplified landscape of dengue host adaptation. The
landscape is shaped by common biophysical and functional constraints
that operate similarly in both hosts, defining the outline of the
fitness landscape. Positive selection of host-specific phenotypes drives
host adaptation. (f) Host adaptation is associated with trade-offs that
form a bottleneck to transmission. This bottleneck is relaxed by
phenotypic redundancy and structural flexibility at key hotspots of
adaptation.

\textbf{{Legends to Extended Data Figures}}

\textbf{Extended Data Figure 1}: Phenotypic characterization of passaged
viral populations. (a) Quantification of virus production, quantified by
focus forming assay for both replicates. (b) Intracellular RNA content
determined by qRT PCR. (c) Efficiency of Plating (EOP) data represented
in the embedding in Fig. 1g. EOP was determined based on comparison to
the adapted cell line.

\textbf{Extended Data Figure 2:} Genotypic characterization of passaged
DENV2 populations. (a) PCA loadings of genetic diversity in sequenced
populations. The first and second components are host specific, while
the third and fourth capture replicate specific differences between the
populations. (b) PC scores of individual allele variants. Each score
represents the contribution of the allele to the specific pattern of
variance captured in the component. Major alleles are highlighted.

\textbf{Extended Data Figure 3}: (a) Box plots of the nine individual
mutation rate estimates obtained for each passaged population,
indicating transitions (``Ts'') and Transversions (``Tv''). (b)
Distribution of mutational fitness effects for all possible alleles in
the population. The bars in the histogram are shaded to show the
proportion of alleles called as Beneficial, Neutral, Deleterious or
Lethal, according to their 95\% CI. (c) Filled Histogram showing the
sequencing depth required to observe alleles of a given fitness class
for all populations. (d) UpSet Plots
\href{https://paperpile.com/c/REZjPf/PH7pe}{\textsuperscript{68}}
comparing shared alleles of individual fitness classes between
experimental sets. These comparisons reveal the stochastic nature of
beneficial mutations, which are largely unique to the individual
populations. Deleterious and lethal mutations act more
deterministically, and exhibit common effects on fitness in the
alternative host environments.

\textbf{Extended Data Figure 4:} (a) \emph{Fitness Wave} representations
of the allele fitness dynamics of all of the experimental populations.
(b) Line plot showing the change in expected fitness of a randomly drawn
allele in the population over time. (c) Line plots showing the mean
number of mutations per genome of each fitness class in the sampled
genomes used to estimate population fitness.

\textbf{Extended Data Movie 1:} Animation of the allele frequencies in
the adapting populations over nine passages. Colors: Orange,
non-synonymous mutations; Green, synonymous mutation; and Grey,
mutations in the UTR.

\textbf{Extended Data}: All data has been deposited and is available at
the persistent URL: \url{https://purl.stanford.edu/gv159td5450}

\textbf{Extended Data Table 1:} Table of fitness estimates, mutation
classifications, and computational statistics for all data presented
here.

\textbf{Extended Data Table 2:} Fitness and disorder prediction, and TM
prediction data used as input for analysis in Fig 6.

\textbf{Extended Data Table 3:} Statistical analyses for enrichment data
presented in Fig. 6.

\textbf{Extended Data Table 4:} Class-specific mutation rate and error
estimates.

\textbf{ACKNOWLEDGEMENTS}

Research reported in this publication was supported by National
Institutes of Health grants AI127447 (JF), AI36178, AI40085, AI091575~
(RA), F32GM113483 (PTD), a DARPA Prophecy Award and fellowships from the
Naito Foundation (ST) and Uehara Memorial Foundation (ST). We thank the
Frydman and Andino labs for discussions and Prof. Marc Feldman and
Dmitri Petrov and their labs for constructive comments on the work.

\textbf{References}

1. \href{http://paperpile.com/b/REZjPf/Nx7yi}{Messina, J. P. \emph{et
al.} The current and future global distribution and population at risk
of dengue. \emph{Nat Microbiol} \textbf{4}, 1508--1515 (2019).}

2. \href{http://paperpile.com/b/REZjPf/9TgsM}{Alto, B. W., Wasik, B. R.,
Morales, N. M. \& Turner, P. E. Stochastic temperatures impede RNA virus
adaptation. \emph{Evolution} \textbf{67}, 969--979 (2013).}

3. \href{http://paperpile.com/b/REZjPf/8nuz1}{Domingo, E. \& Perales, C.
Virus Evolution. \emph{eLS} (2014).
doi:}\href{http://dx.doi.org/10.1002/9780470015902.a0000436.pub3}{10.1002/9780470015902.a0000436.pub3}

4. \href{http://paperpile.com/b/REZjPf/tRYDR}{Sanjuán, R. Viral Mutation
Rates. \emph{Virus Evolution: Current Research and Future Directions}
1--28 (2016).
doi:}\href{http://dx.doi.org/10.21775/9781910190234.01}{10.21775/9781910190234.01}

5. \href{http://paperpile.com/b/REZjPf/L1Afo}{Dolan, P. T., Whitfield,
Z. J. \& Andino, R. Mechanisms and Concepts in RNA Virus Population
Dynamics and Evolution. \emph{Annu Rev Virol} \textbf{5}, 69--92
(2018).}

6. \href{http://paperpile.com/b/REZjPf/mlDyr}{Sessions, O. M. \emph{et
al.} Analysis of Dengue Virus Genetic Diversity during Human and
Mosquito Infection Reveals Genetic Constraints. \emph{PLoS Negl. Trop.
Dis.} \textbf{9}, e0004044 (2015).}

7. \href{http://paperpile.com/b/REZjPf/r0H6}{Villordo, S. M.,
Filomatori, C. V., Sánchez-Vargas, I., Blair, C. D. \& Gamarnik, A. V.
Dengue virus RNA structure specialization facilitates host adaptation.
\emph{PLoS Pathog.} \textbf{11}, e1004604 (2015).}

8. \href{http://paperpile.com/b/REZjPf/erQn1}{Pompon, J. \emph{et al.}
Dengue subgenomic flaviviral RNA disrupts immunity in mosquito salivary
glands to increase virus transmission. \emph{PLoS Pathog.} \textbf{13},
e1006535 (2017).}

9. \href{http://paperpile.com/b/REZjPf/mYIl}{Filomatori, C. V. \emph{et
al.} Dengue virus genomic variation associated with mosquito adaptation
defines the pattern of viral non-coding RNAs and fitness in human cells.
\emph{PLoS Pathog.} \textbf{13}, e1006265 (2017).}

10. \href{http://paperpile.com/b/REZjPf/uzXme}{Stapleford, K. A.
\emph{et al.} Emergence and transmission of arbovirus evolutionary
intermediates with epidemic potential. \emph{Cell Host Microbe}
\textbf{15}, 706--716 (2014).}

11. \href{http://paperpile.com/b/REZjPf/XqhpA}{Coffey, L. L. \&
Vignuzzi, M. Host alternation of chikungunya virus increases fitness
while restricting population diversity and adaptability to novel
selective pressures. \emph{J. Virol.} \textbf{85}, 1025--1035 (2011).}

12. \href{http://paperpile.com/b/REZjPf/fySKT}{Lauring, A. S. \& Andino,
R. Quasispecies theory and the behavior of RNA viruses. \emph{PLoS
Pathog.} \textbf{6}, e1001005 (2010).}

13. \href{http://paperpile.com/b/REZjPf/kVUH4}{Andino, R. \& Domingo, E.
Viral quasispecies. \emph{Virology} \textbf{479-480}, 46--51 (2015).}

14. \href{http://paperpile.com/b/REZjPf/ETeaC}{Vignuzzi, M., Stone, J.
K., Arnold, J. J., Cameron, C. E. \& Andino, R. Quasispecies diversity
determines pathogenesis through cooperative interactions in a viral
population. \emph{Nature} \textbf{439}, 344--348 (2006).}

15. \href{http://paperpile.com/b/REZjPf/lp4YG}{Xue, K. S., Greninger, A.
L., Pérez-Osorio, A. \& Bloom, J. D. Cooperating H3N2 Influenza Virus
Variants Are Not Detectable in Primary Clinical Samples. \emph{mSphere}
\textbf{3}, (2018).}

16. \href{http://paperpile.com/b/REZjPf/iU0v6}{Xue, K. S., Hooper, K.
A., Ollodart, A. R., Dingens, A. S. \& Bloom, J. D. Cooperation between
distinct viral variants promotes growth of H3N2 influenza in cell
culture. \emph{Elife} \textbf{5}, e13974 (2016).}

17. \href{http://paperpile.com/b/REZjPf/f5vgZ}{Wilke, C. O., Forster, R.
\& Novella, I. S. Quasispecies in Time-Dependent Environments.
\emph{Current Topics in Microbiology and Immunology} 33--50
doi:}\href{http://dx.doi.org/10.1007/3-540-26397-7_2}{10.1007/3-540-26397-7\_2}

18. \href{http://paperpile.com/b/REZjPf/zRVIk}{Grad, Y. H. \emph{et al.}
Within-host whole-genome deep sequencing and diversity analysis of human
respiratory syncytial virus infection reveals dynamics of genomic
diversity in the absence and presence of immune pressure. \emph{J.
Virol.} \textbf{88}, 7286--7293 (2014).}

19. \href{http://paperpile.com/b/REZjPf/vDnrd}{Bordería, A. V. \emph{et
al.} Group Selection and Contribution of Minority Variants during Virus
Adaptation Determines Virus Fitness and Phenotype. \emph{PLoS Pathog.}
\textbf{11}, e1004838 (2015).}

20. \href{http://paperpile.com/b/REZjPf/bczFl}{Acevedo, A., Brodsky, L.
\& Andino, R. Mutational and fitness landscapes of an RNA virus revealed
through population sequencing. \emph{Nature} \textbf{505}, 686--690
(2014).}

21. \href{http://paperpile.com/b/REZjPf/CObqp}{Acevedo, A. \& Andino, R.
Library preparation for highly accurate population sequencing of RNA
viruses. \emph{Nat. Protoc.} \textbf{9}, 1760--1769 (2014).}

22. \href{http://paperpile.com/b/REZjPf/Bai6p}{Johnson, A. J.,
Guirakhoo, F. \& Roehrig, J. T. The Envelope Glycoproteins of Dengue 1
and Dengue 2 Viruses Grown in Mosquito Cells Differ in Their Utilization
of Potential Glycosylation Sites. \emph{Virology} \textbf{203}, 241--249
(1994).}

23. \href{http://paperpile.com/b/REZjPf/a4AV4}{Novella, I. S. \emph{et
al.} Extreme fitness differences in mammalian and insect hosts after
continuous replication of vesicular stomatitis virus in sandfly cells.
\emph{J. Virol.} \textbf{69}, 6805--6809 (1995).}

24. \href{http://paperpile.com/b/REZjPf/knFQ3}{Villordo, S. M. \&
Gamarnik, A. V. Differential RNA Sequence Requirement for Dengue Virus
Replication in Mosquito and Mammalian Cells. \emph{Journal of Virology}
\textbf{87}, 9365--9372 (2013).}

25. \href{http://paperpile.com/b/REZjPf/Mq62m}{Vasilakis, N. \emph{et
al.} Mosquitoes put the brake on arbovirus evolution: experimental
evolution reveals slower mutation accumulation in mosquito than
vertebrate cells. \emph{PLoS Pathog.} \textbf{5}, e1000467 (2009).}

26. \href{http://paperpile.com/b/REZjPf/4eMRo}{Byk, L. A. \& Gamarnik,
A. V. Properties and Functions of the Dengue Virus Capsid Protein.
\emph{Annual Review of Virology} \textbf{3}, 263--281 (2016).}

27. \href{http://paperpile.com/b/REZjPf/YjlCW}{Greene, I. P. \emph{et
al.} Effect of alternating passage on adaptation of sindbis virus to
vertebrate and invertebrate cells. \emph{J. Virol.} \textbf{79},
14253--14260 (2005).}

28. \href{http://paperpile.com/b/REZjPf/jN2vZ}{Smith, D. J. \emph{et
al.} Mapping the antigenic and genetic evolution of influenza virus.
\emph{Science} \textbf{305}, 371--376 (2004).}

29. \href{http://paperpile.com/b/REZjPf/jDAqh}{Whitfield, Z. J. \&
Andino, R. Characterization of Viral Populations by Using Circular
Sequencing. \emph{J. Virol.} \textbf{90}, 8950--8953 (2016).}

30. \href{http://paperpile.com/b/REZjPf/Clw9O}{Wilke, C. O. Quasispecies
theory in the context of population genetics. \emph{BMC Evol. Biol.}
\textbf{5}, 44 (2005).}

31. \href{http://paperpile.com/b/REZjPf/rZbzc}{Sardanyés, J., Elena, S.
F. \& Solé, R. V. Simple quasispecies models for the
survival-of-the-flattest effect: The role of space. \emph{J. Theor.
Biol.} \textbf{250}, 560--568 (2008).}

32. \href{http://paperpile.com/b/REZjPf/M9gyT}{Milewska, A. \emph{et
al.} APOBEC3-mediated restriction of RNA virus replication. \emph{Sci.
Rep.} \textbf{8}, 5960 (2018).}

33. \href{http://paperpile.com/b/REZjPf/I2u2G}{Drake, J. W. \& Holland,
J. J. Mutation rates among RNA viruses. \emph{Proc. Natl. Acad. Sci. U.
S. A.} \textbf{96}, 13910--13913 (1999).}

34. \href{http://paperpile.com/b/REZjPf/delW7}{Villordo, S. M., Alvarez,
D. E. \& Gamarnik, A. V. A balance between circular and linear forms of
the dengue virus genome is crucial for viral replication. \emph{RNA}
\textbf{16}, 2325--2335 (2010).}

35. \href{http://paperpile.com/b/REZjPf/4f1JN}{Lodeiro, M. F.,
Filomatori, C. V. \& Gamarnik, A. V. Structural and functional studies
of the promoter element for dengue virus RNA replication. \emph{J.
Virol.} \textbf{83}, 993--1008 (2009).}

36. \href{http://paperpile.com/b/REZjPf/EVbva}{Pechmann, S. \& Frydman,
J. Interplay between chaperones and protein disorder promotes the
evolution of protein networks. \emph{PLoS Comput. Biol.} \textbf{10},
e1003674 (2014).}

37. \href{http://paperpile.com/b/REZjPf/rysMZ}{Kimura, M. On the change
of population fitness by natural selection. \emph{Heredity} \textbf{12},
145--167 (1958).}

38. \href{http://paperpile.com/b/REZjPf/WYtJW}{Orr, H. A. \& Allen Orr,
H. Fitness and its role in evolutionary genetics. \emph{Nature Reviews
Genetics} \textbf{10}, 531--539 (2009).}

39. \href{http://paperpile.com/b/REZjPf/hSNe}{Mondotte, J. A., Lozach,
P.-Y., Amara, A. \& Gamarnik, A. V. Essential role of dengue virus
envelope protein N glycosylation at asparagine-67 during viral
propagation. \emph{J. Virol.} \textbf{81}, 7136--7148 (2007).}

40. \href{http://paperpile.com/b/REZjPf/gTFia}{Chung, C.-Y., Majewska,
N. I., Wang, Q., Paul, J. T. \& Betenbaugh, M. J. SnapShot:
N-Glycosylation Processing Pathways across Kingdoms. \emph{Cell}
\textbf{171}, 258--258.e1 (2017).}

41. \href{http://paperpile.com/b/REZjPf/JUNqO}{Yap, S. S. L.,
Nguyen-Khuong, T., Rudd, P. M. \& Alonso, S. Dengue Virus Glycosylation:
What Do We Know? \emph{Front. Microbiol.} \textbf{8}, 1415 (2017).}

42. \href{http://paperpile.com/b/REZjPf/hrhvV}{Sirohi, D. \emph{et al.}
The 3.8 Å resolution cryo-EM structure of Zika virus. \emph{Science}
\textbf{352}, 467--470 (2016).}

43. \href{http://paperpile.com/b/REZjPf/Zs1LR}{Chapman, E. G., Moon, S.
L., Wilusz, J. \& Kieft, J. S. RNA structures that resist degradation by
Xrn1 produce a pathogenic Dengue virus RNA. \emph{Elife} \textbf{3},
e01892 (2014).}

44. \href{http://paperpile.com/b/REZjPf/b9GgS}{Chapman, E. G. \emph{et
al.} The structural basis of pathogenic subgenomic flavivirus RNA
(sfRNA) production. \emph{Science} \textbf{344}, 307--310 (2014).}

45. \href{http://paperpile.com/b/REZjPf/o9YCc}{Akiyama, B. M., Eiler, D.
\& Kieft, J. S. Structured RNAs that evade or confound exonucleases:
function follows form. \emph{Curr. Opin. Struct. Biol.} \textbf{36},
40--47 (2016).}

46. \href{http://paperpile.com/b/REZjPf/XAKft}{Opekarová, M. \& Tanner,
W. Specific lipid requirements of membrane proteins-\/-a putative
bottleneck in heterologous expression. \emph{Biochim. Biophys. Acta}
\textbf{1610}, 11--22 (2003).}

47. \href{http://paperpile.com/b/REZjPf/zRdX7}{Hafer, A., Whittlesey,
R., Brown, D. T. \& Hernandez, R. Differential incorporation of
cholesterol by Sindbis virus grown in mammalian or insect cells.
\emph{J. Virol.} \textbf{83}, 9113--9121 (2009).}

48. \href{http://paperpile.com/b/REZjPf/mhPR}{Charon, J. \emph{et al.}
First Experimental Assessment of Protein Intrinsic Disorder Involvement
in an RNA Virus Natural Adaptive Process. \emph{Mol. Biol. Evol.}
\textbf{35}, 38--49 (2018).}

49. \href{http://paperpile.com/b/REZjPf/KHg7}{Goh, G. K.-M., Dunker, A.
K. \& Uversky, V. N. Correlating Flavivirus virulence and levels of
intrinsic disorder in shell proteins: protective roles vs. immune
evasion. \emph{Mol. Biosyst.} \textbf{12}, 1881--1891 (2016).}

50. \href{http://paperpile.com/b/REZjPf/KuSMk}{Shoval, O. \emph{et al.}
Evolutionary trade-offs, Pareto optimality, and the geometry of
phenotype space. \emph{Science} \textbf{336}, 1157--1160 (2012).}

51. \href{http://paperpile.com/b/REZjPf/gqelx}{Tendler, A., Mayo, A. \&
Alon, U. Evolutionary tradeoffs, Pareto optimality and the morphology of
ammonite shells. \emph{BMC Syst. Biol.} \textbf{9}, 12 (2015).}

52. \href{http://paperpile.com/b/REZjPf/QZFTJ}{Bourg, S., Jacob, L.,
Menu, F. \& Rajon, E. Hormonal pleiotropy and the evolution of
allocation trade-offs. \emph{Evolution} \textbf{73}, 661--674 (2019).}

53. \href{http://paperpile.com/b/REZjPf/pUdK}{Gitlin, L., Hagai, T.,
LaBarbera, A., Solovey, M. \& Andino, R. Rapid evolution of virus
sequences in intrinsically disordered protein regions. \emph{PLoS
Pathog.} \textbf{10}, e1004529 (2014).}

54. \href{http://paperpile.com/b/REZjPf/py7tY}{Taguwa, S. \emph{et al.}
Defining Hsp70 Subnetworks in Dengue Virus Replication Reveals Key
Vulnerability in Flavivirus Infection. \emph{Cell} \textbf{163},
1108--1123 (2015).}

55. \href{http://paperpile.com/b/REZjPf/TJLiW}{Therneau, T. deming:
Deming, Thiel-Sen and Passing-Bablock Regression. \emph{R package
version} 1--0 (2014).}

56. \href{http://paperpile.com/b/REZjPf/xDHhy}{Sen, P. K. Robustness of
Some Nonparametric Procedures in Linear Models. \emph{Ann. Math. Stat.}
\textbf{39}, 1913--1922 (1968).}

57. \href{http://paperpile.com/b/REZjPf/uGyfe}{R Core Team. R: A
Language and Environment for Statistical Computing. (2015).}

58. \href{http://paperpile.com/b/REZjPf/V25w7}{Jombart, T. adegenet: a R
package for the multivariate analysis of genetic markers.
\emph{Bioinformatics} \textbf{24}, 1403--1405 (2008).}

59. \href{http://paperpile.com/b/REZjPf/TEfJA}{Kamvar, Z. N., Tabima, J.
F. \& Grünwald, N. J. Poppr: an R package for genetic analysis of
populations with clonal, partially clonal, and/or sexual reproduction.
\emph{PeerJ} \textbf{2}, e281 (2014).}

60. \href{http://paperpile.com/b/REZjPf/0m4TS}{Torgerson, W. S. PsycNET.
(1958). Available at:}
\url{http://psycnet.apa.org/psycinfo/1959-07320-000.}
\href{http://paperpile.com/b/REZjPf/0m4TS}{(Accessed: 19th March 2019)}

61. \href{http://paperpile.com/b/REZjPf/8bj1M}{Reynolds, J., Weir, B. S.
\& Cockerham, C. C. Estimation of the coancestry coefficient: basis for
a short-term genetic distance. \emph{Genetics} \textbf{105}, 767--779
(1983).}

62. \href{http://paperpile.com/b/REZjPf/gUxkc}{de Leeuw, J. \& Mair, P.
Multidimensional Scaling Using Majorization: SMACOF in R. (2011).}

63. \href{http://paperpile.com/b/REZjPf/dJiR3}{{[}PDF{]}Package `smacof'
- CRAN.}

64. \href{http://paperpile.com/b/REZjPf/pdAby}{HOFMANN \& K. TMbase-a
database of membrane spanning proteins segments. \emph{Biol. Chem. Hoppe
Seyler} \textbf{374}, 166 (1993).}

65. \href{http://paperpile.com/b/REZjPf/GYlhT}{Mészáros, B., Erdos, G.
\& Dosztányi, Z. IUPred2A: context-dependent prediction of protein
disorder as a function of redox state and protein binding. \emph{Nucleic
Acids Res.} \textbf{46}, W329--W337 (2018).}

66. \href{http://paperpile.com/b/REZjPf/lMDMS}{Hochberg, Y. \&
Benjamini, Y. More powerful procedures for multiple significance
testing. \emph{Statistics in Medicine} \textbf{9}, 811--818 (1990).}

67. \href{http://paperpile.com/b/REZjPf/gK8Xk}{Thompson, J. D., Higgins,
D. G. \& Gibson, T. J. CLUSTAL W: improving the sensitivity of
progressive multiple sequence alignment through sequence weighting,
position-specific gap penalties and weight matrix choice. \emph{Nucleic
Acids Research} \textbf{22}, 4673--4680 (1994).}

68. \href{http://paperpile.com/b/REZjPf/PH7pe}{Lex, A., Gehlenborg, N.,
Strobelt, H., Vuillemot, R. \& Pfister, H. UpSet: Visualization of
Intersecting Sets. \emph{IEEE Trans. Vis. Comput. Graph.} \textbf{20},
1983--1992 (2014).}

\end{document}
